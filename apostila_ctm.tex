% Book document class
\documentclass{book}
\usepackage[a4paper,left=1.5cm,right=1.5cm,top=2.5cm,bottom=1.5cm]{geometry}
\usepackage[brazil]{babel}
\usepackage{babelbib}
\usepackage[utf8]{inputenc}
\usepackage{amssymb}
\usepackage{amsmath}
\usepackage{amsthm}
\usepackage{booktabs}
\usepackage{textgreek}

\title{Metrologia Elétrica}
\author{Gean Marcos Geronymo}
\date{\today}

\begin{document}

\maketitle

\frontmatter
%\chapter{Dedication}
%\chapter{Copyright}
%\chapter{Acknowledgements}

\tableofcontents

\listoffigures
\listoftables

\mainmatter
\part{Noções de Metrologia Elétrica}
% segundo a ementa preliminar são 9 aulas para a parte I - noções de Metrologia Elétrica


\chapter{Introdução}
% falar sobre a estrutura da Diele, laboratórios, etc.

\section{Sistema Internacional de Unidades - SI}
O Sistema Internacional de Unidades (SI)~\cite{si-brochure,si-ptbr} é a base da metrologia moderna. Antes da redefinição em 2019, o SI era definido através de sete unidades de base, a partir das quais as unidades derivadas eram obtidas através de operações matemáticas de multiplicação e potenciação, combinando as unidades de base. Após a redefinição do SI, com a fixação dos valores numéricos de sete constantes definidoras, em princípio essa distinção não é mais necessária, já que tanto as unidades de base quanto as unidades derivadas podem ser expressas diretamente em função das constantes definidoras. Entretanto, o conceito de unidades base e unidades derivadas foi mantido por razões históricas e por compatibilidade com normas estabelecidas.

% falar das constantes definidoras

A definição formal do SI desde 2019~\cite{si-brochure,si-ptbr} é baseada em sete constantes definidoras, apresentadas na \tablename~\ref{tab:si_cte}. O valor numérico das constantes definidoras possui incerteza nula. As unidades das constantes definidoras, hertz (Hz), joule (J), coulomb (C), lumen (lm) e watt (W) são relacionadas às unidades de base (ver \tablename~\ref{tab:si_base}) segundo (s), metro (m), kilograma (kg), ampere (A), kelvin (K), mole (mol) e candela (cd) de acordo com:
\begin{itemize}
\item Hz = s\textsuperscript{-1}
\item J = kg m\textsuperscript{2}s\textsuperscript{-2}
\item C = A s
\item lm = cd m\textsuperscript{2} m\textsubscript{-2} = cd sr (onde sr é o ângulo sólido)
  \item W = m\textsuperscript{2}s\textsuperscript{-3}
\end{itemize}

\begin{table}[!ht]
\centering
\caption{Constantes definidoras do SI.}
\begin{tabular}{cccc}
  \toprule[.8pt]
  Constante & Símbolo & Valor numérico & Unidade \\
  \midrule[.5pt]
  frequência de transição hiperfina do Cs & $\Delta \nu_{Cs}$ & 9 192 631 770 & Hz \\
  velocidade da luz no vácuo & $c$ & 299 792 458 & m s\textsuperscript{-1} \\
  constante de Planck & $h$ & $6,626~070~15 \times 10^{-34} $ & J s \\
  carga elementar & $e$ & $1,602~176~634 \times 10^{-19}$ & C \\
  constante de Boltzmann & $k$ & $1,380~649 \times 10^{-23} $ & J K\textsuperscript{-1} \\
  constante de Avogadro & $N_A$ & $6,022~140~76 \times 10^{23} $ & mol \textsuperscript{-1} \\
  eficácia luminosa & $K_{cd}$ & 683 & lm W\textsuperscript{-1} \\
  \bottomrule[.8pt]
\end{tabular}
\label{tab:si_cte}
\end{table}

% falar das unidades de base

\begin{table}[!ht]
\centering
\caption{Unidades de base do SI.}
\begin{tabular}{ccc}
  \toprule[.8pt]
  Grandeza & Unidade & Símbolo \\
  \midrule[.5pt]
  tempo & segundo & s \\
  comprimento & metro & m \\
  massa & kilograma & kg \\
  corrente elétrica & ampere & A \\
  temperatura termodinâmica & kelvin & K \\
  quantidade de matéria & mole & mol \\
  intensidade luminosa & candela & cd \\
  \bottomrule[.8pt]
\end{tabular}
\label{tab:si_base}
\end{table}



% falar das unidades derivadas -> unidades elétricas


A \tablename~\ref{tab:mult_factors} apresenta os fatores de multiplicação das unidades do SI. Os fatores de multiplicação mais utilizados em Metrologia Elétrica estão destacados em negrito.

\begin{table}[!ht]
\centering
\caption{Fatores de multiplicação.}
\begin{tabular}{ccc}
  \toprule[.8pt]
  \textbf{Fator} & \textbf{Nome} & \textbf{Símbolo} \\
  \midrule[.5pt]
  10\textsuperscript{24} & yotta & Y \\
  10\textsuperscript{21} & zeta & Z \\
  10\textsuperscript{18} & exa & E \\
  10\textsuperscript{15} & peta & P \\
  \textbf{10\textsuperscript{12}} & \textbf{tera} & \textbf{T} \\
  \textbf{10\textsuperscript{9}} & \textbf{giga} & \textbf{G} \\
  \textbf{10\textsuperscript{6}} & \textbf{mega} & \textbf{M} \\
  \textbf{10\textsuperscript{3}} & \textbf{kilo} & \textbf{k} \\
  10\textsuperscript{2} & hecto & h \\
  10\textsuperscript{1} & deka & da \\
  10\textsuperscript{-1} & deci & d \\
  10\textsuperscript{-2} & centi & c \\
  \textbf{10\textsuperscript{-3}} & \textbf{mili} & \textbf{m} \\
  \textbf{10\textsuperscript{-6}} & \textbf{micro} & \textbf{\textmu} \\
  \textbf{10\textsuperscript{-9}} & \textbf{nano} & \textbf{n} \\
  \textbf{10\textsuperscript{-12}} & \textbf{pico} & \textbf{p} \\
  10\textsuperscript{-15} & femto & f \\
  10\textsuperscript{-18} & atto & a \\
  10\textsuperscript{-21} & zepto & z \\
  10\textsuperscript{-24} & yocto & y \\
  \bottomrule[.8pt]
\end{tabular}
\label{tab:mult_factors}
\end{table}

\chapter{Padronização}
\section{Resistência Elétrica}

Segundo a Lei de Ohm, a resistência elétrica $\mathrm{R}$ de um condutor, em ohms ($\mathrm{\Omega}$), é a razão da diferença de potencial $\mathrm{V}$, em volts (V), aplicada a esse condutor, e da corrente elétrica $\mathrm{I}$ em amperes (A) fluindo através do mesmo condutor:
\begin{equation}
\mathrm{R} = \dfrac{\mathrm{V}}{\mathrm{I}}
\end{equation}

A resistência elétrica também pode ser determinada em função das características do material, através da equação:
\begin{equation}
  \mathrm{R} = \rho \cdot \dfrac{l}{A}
\end{equation}
onde $\rho$ é a resistividade elétrica do condutor, em ohm-metro ($\mathrm{\Omega} \cdot \mathrm{m}$), $l$ é o comprimento do condutor, em metros, e $A$ é área da seção transversal do condutor, em metros quadrados.

Entretanto, experimentos mais precisos mostram que a resistência elétrica também é função da temperatura e até mesmo da presença de tensão mecânica no condutor~\cite{fluke}. Com base nesses fenômenos, foram desenvolvidos trandutores\footnote{Transdutor é um dispositivo utilizado em conversão de energia de uma natureza para outra. São muito utilizados para converter grandezas como posição, velocidade, temperatura, luz, pressão, etc. em sinais elétricos.} para a medição de temperatura, pequenos deslocamentos, pressão em líquidos, dentre outros. Além disso, a corrente elétrica, que é uma grandeza de base do SI, normalmente é medida através da diferença de potencial em um resistor conhecido. De fato, uma grande parte das grandezas elétricas é medida usando métodos que envolvem de alguma forma a medição de resistência.

\section{Capacitância e Indutância}

\subsection{Circuitos em corrente alternada}

\section{Tensão Elétrica}
\section{Transferência AC-DC}
\chapter{Calibração de Medidores e Fontes}


%\appendix
%\chapter{First and only appendix}

\backmatter
%\chapter{Bibliografia}
% incluir fluke calibration philosophy in practice
%\chapter{Other titles in this collection}

\bibliographystyle{ieeetr}
\bibliography{bibliografia}

\end{document}