% Book document class
\documentclass{book}
\usepackage[a4paper,left=1.5cm,right=1.5cm,top=2.5cm,bottom=1.5cm]{geometry}
\usepackage[brazil]{babel}
\usepackage[utf8]{inputenc}
\usepackage{amssymb}
\usepackage{amsmath}
\usepackage{amsthm}

\title{Metrologia Elétrica}
\author{Gean Marcos Geronymo}
\date{\today}

\begin{document}

\maketitle

\frontmatter
\chapter{Dedication}
\chapter{Copyright}
\chapter{Acknowledgements}

\tableofcontents

\listoffigures
\listoftables

\mainmatter
\part{Noções de Metrologia Elétrica}
% segundo a ementa preliminar são 9 aulas para a parte I - noções de Metrologia Elétrica


\chapter{Introdução}
% falar sobre a estrutura da Diele, laboratórios, etc.
\chapter{Padronização}
\section{Resistência Elétrica}

Segundo a Lei de Ohm, a resistência elétrica $\mathrm{R}$ de um condutor, em ohms ($\mathrm{\Omega}$), é a razão da diferença de potencial $\mathrm{V}$, em volts (V), aplicada a esse condutor, e da corrente elétrica $\mathrm{I}$ em amperes (A) fluindo através do mesmo condutor:
\begin{equation}
\mathrm{R} = \dfrac{\mathrm{V}}{\mathrm{I}}
\end{equation}

A resistência elétrica também pode ser determinada em função das características do material, através da equação:
\begin{equation}
  \mathrm{R} = \rho \cdot \dfrac{l}{A}
\end{equation}
onde $\rho$ é a resistividade elétrica do condutor, em ohm-metro ($\mathrm{\Omega} \cdot \mathrm{m}$), $l$ é o comprimento do condutor, em metros, e $A$ é área da seção transversal do condutor, em metros quadrados.

Entretanto, experimentos mais precisos mostram que a resistência elétrica também é função da temperatura e até mesmo da presença de tensão mecânica no condutor. Com base nesses fenômenos, foram desenvolvidos trandutores\footnote{Transdutor é um dispositivo utilizado em conversão de energia de uma natureza para outra. São muito utilizados para converter grandezas como posição, velocidade, temperatura, luz, pressão, etc. em sinais elétricos.} para a medição de temperatura, pequenos deslocamentos, pressão em líquidos, dentre outros. Além disso, a corrente elétrica, que é uma grandeza de base do SI, normalmente é medida através da diferença de potencial em um resistor conhecido. De fato, uma grande parte das grandezas elétricas é medida usando métodos que envolvem de alguma forma a medição de resistência.

\section{Capacitância e Indutância}
\section{Tensão Elétrica}
\section{Transferência AC-DC}
\chapter{Calibração de Medidores e Fontes}


\appendix
\chapter{First and only appendix}

\backmatter
\chapter{Bibliography}
\chapter{Other titles in this collection}

\end{document}